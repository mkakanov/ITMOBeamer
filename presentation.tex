%!TEX output_directory = tmp
\documentclass[aspectratio=169]{beamer}

\usepackage{ITMOtemplate}

\usepackage{hyperref}
\usepackage{graphicx}
\usepackage{tikz}
\usepackage{graphics}
\usepackage{epsfig}
\usepackage{mathptmx}
\usepackage{times}
\usepackage{amsmath}
\usepackage{amssymb}
\usepackage{epstopdf}
\usepackage{xcolor}
% \usepackage{color, soul}
% \usepackage{soul} 

% \makeatletter
% \let\HL\hl
% \renewcommand\hl{%
%   \let\set@color\beamerorig@set@color
%   \let\reset@color\beamerorig@reset@color
%   \HL}
% \makeatother
% \setstcolor{red} 
% \sethlcolor{ITMOred} 

\DeclareMathOperator{\sat}{sat}
\DeclareMathOperator{\diag}{diag}

\renewcommand{\Im}{Im}
% \usepackage{fontspec}
% \setmainfont{Calibri}

\graphicspath{{figs/}}

\title{ITMO Template Example}

\author[Author, Another]{
  {\bf First~A.}\inst{1} \and 
  Second~A.\inst{2}
}

\institute[ITMO University] % (optional, but mostly needed)
{
\scriptsize
\begin{minipage}[t]{0.4\textwidth}
    \centering{\inst{1}%
  Faculty/Department \\ 
  ITMO University}
\end{minipage}\quad
\begin{minipage}[t]{0.4\textwidth}
    \centering{\inst{2}%
  Faculty/Department \\ 
  ITMO University}
\end{minipage}
 }

\date[Ocasion]{Date\\Ocasion}

\begin{document}


\begin{frame}[plain]
	\titlepage
\end{frame}


\begin{frame}\frametitle{Table of Contents}
  \tableofcontents
\end{frame}

\section{Mathematics}
\subsection{Theorem}


\begin{frame}{Mathematics}
    \begin{theorem}[Fermat's little theorem]
        For a prime~\(p\) and \(a \in \mathbb{Z}\) it holds that \(a^p \equiv a \pmod{p}\).
    \end{theorem}

    \begin{proof}
        The invertible elements in a field form a group under multiplication.
        In particular, the elements
        \begin{equation*}
            1, 2, \ldots, p - 1 \in \mathbb{Z}_p
        \end{equation*}
        form a group under multiplication modulo~\(p\).
        This is a group of order \(p - 1\).
        For \(a \in \mathbb{Z}_p\) and \(a \neq 0\) we thus get \(a^{p-1} = 1 \in \mathbb{Z}_p\).
        The claim follows.
    \end{proof}
\end{frame}


\subsection{Example}


\begin{frame}{Mathematics}
    \begin{example}
        The function \(\phi \colon \mathbb{R} \to \mathbb{R}\) given by \(\phi(x) = 2x\) is continuous at the point \(x = \alpha\),
        because if \(\epsilon > 0\) and \(x \in \mathbb{R}\) is such that \(\lvert x - \alpha \rvert < \delta = \frac{\epsilon}{2}\),
        then
        \begin{equation*}
            \lvert \phi(x) - \phi(\alpha)\rvert = 2\lvert x - \alpha \rvert < 2\delta = \epsilon.
        \end{equation*}
    \end{example}
\end{frame}


\section{Highlighting}


\begin{frame}{Highlighting}
    Sometimes it is useful to \boxalert{highlight} certain words in the text.

    \begin{alertblock}{Important message}
        If a lot of text should be \alert{highlighted}, it is a good idea to put it in a box.
    \end{alertblock}
    

    It is easy to match the \structure{colour theme}.
\end{frame}


\section{Lists}


\begin{frame}{Lists}
    \begin{fancylist}
        \item Fancy lists are marked with a number inside a circle.
    \end{fancylist}

    \begin{itemize}
        \item
        Bullet lists are marked with a red box.
    \end{itemize}

    \begin{enumerate}
        \item
        Numbered lists are marked with a white number inside a red box.
    \end{enumerate}

    \begin{example}
        \begin{itemize}
            \item
            Lists change colour after the environment.
        \end{itemize}
    \end{example}
\end{frame}


\section{Effects}


\begin{frame}{Effects}
    \begin{columns}[onlytextwidth]
        \begin{column}{0.49\textwidth}
            \begin{enumerate}[<+-|alert@+>]
                \item
                Effects that control

                \item
                when text is displayed

                \item
                are specified with <> and a list of slides.
            \end{enumerate}

            \begin{theorem}<2>
                This theorem is only visible on slide number 2.
            \end{theorem}
        \end{column}
        \begin{column}{0.49\textwidth}
            Use \textbf<2->{textblock} for arbitrary placement of objects.

            \pause
            \medskip

            It creates a box
            with the specified width (here in a percentage of the slide's width)
            and upper left corner at the specified coordinate (x, y)
            (here x is a percentage of width and y a percentage of height).
        \end{column}
    \end{columns}
    
    % \begin{textblock}{0.3}(0.45, 0.55)
    %     \includegraphics<1, 3>[width = \textwidth]{UiB-images/UiB-emblem}
    % \end{textblock}
\end{frame}


\section{References}

\end{document}